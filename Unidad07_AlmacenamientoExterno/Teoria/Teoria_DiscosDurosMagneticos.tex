% Created 2019-11-19 mar 18:19
% Intended LaTeX compiler: pdflatex
\documentclass[11pt]{article}
\usepackage[utf8]{inputenc}
\usepackage[T1]{fontenc}
\usepackage{graphicx}
\usepackage{grffile}
\usepackage{longtable}
\usepackage{wrapfig}
\usepackage{rotating}
\usepackage[normalem]{ulem}
\usepackage{amsmath}
\usepackage{textcomp}
\usepackage{amssymb}
\usepackage{capt-of}
\usepackage{hyperref}
\usepackage[newfloat]{minted}
\hypersetup{colorlinks=true,linkcolor=black}
\author{Angel Berlanas}
\date{\today}
\title{UD07 - Discos Duros - Magnéticos}
\hypersetup{
 pdfauthor={Angel Berlanas},
 pdftitle={UD07 - Discos Duros - Magnéticos},
 pdfkeywords={},
 pdfsubject={},
 pdfcreator={Emacs 25.2.2 (Org mode 9.2.5)}, 
 pdflang={English}}
\begin{document}

\maketitle
\tableofcontents



\section{Funcionamiento}
\label{sec:org798a161}

El disco es en realidad una pila de discos llamados platos que almacenan la informa-
ción magnéticamente. Los diferentes platos que forman el disco giran a una velocidad
constante y no cesan mientras el ordenador está encendido. Cada cara del plato tiene
asignado uno de los cabezales de lectura/escritura.

Las acciones que ejecuta el disco duro en una operación de lectura son:

\begin{enumerate}
\item Desplazar los cabezales de lectura/escritura hasta el lugar donde empiezan los datos.
\item Esperar a que el primer dato llegue donde están los cabezales.
\item Leer el dato con el cabezal.
\end{enumerate}

La operación de escritura es similar. El funcionamiento teórico es sencillo, pero en la
realidad es mucho más complejo, ya que entran en juego el procesador, la controladora
de discos, la BIOS/EFI, el sistema operativo, la memoria RAM y el propio disco,
con la caché, etc.

\section{Cabezas, cilindros y sectores}
\label{sec:org9a712f6}

Para organizar los datos en un disco duro se utilizan tres parámetros, que definen la
estructura física del disco: \emph{cabeza, cilindro y sector}.

\subsection{Cabezas}
\label{sec:orge0d26dd}

Cada una de las caras o cabezas del disco se divide en anillos concéntricos denominados pistas (tracks), que es donde se graban los datos.

\subsection{Cilindro (cylinder).}
\label{sec:org60e2508}

Formados por todas las pistas accesibles en una posición de los cabezales. Se
utiliza este término para referirse a la misma pista de todos los discos de la pila.

\subsection{Sectores}
\label{sec:org537f0b5}

Cada pista se encuentra dividida en tramos o arcos iguales que permiten la grabación
de bloques de bytes (normalmente, 512 B). Cada uno de estos tramos se llama sector. Los sectores
son las unidades mínimas de información que pueden leerse o escribirse en el disco duro.

\section{Estructura lógica}
\label{sec:org2fc95ce}

La estructura lógica de un disco duro es la forma en la que se organiza la información
que contiene. Está formada por:

\subsection{El sector de arranque (master boot record)}
\label{sec:orgac4bb00}

Es el primer sector de todo el disco duro: cabeza 0, cilindro 0 y
sector 1. En él se almacena la tabla de particiones, que contiene información acerca del inicio y el fin de cada partición, y un pequeño programa
llamado master boot, que es el encargado de leer la tabla de particiones y ceder el
control al sector de arranque de la partición activa, desde la que arranca el PC.

\subsection{El espacio particionado}
\label{sec:org6b7cc9a}

   Es el espacio de disco duro que ha sido asignado a alguna partición. Las particiones se definen por cilindros. Cada partición tiene un nombre; en
los sistemas Microsoft llevan asociada una letra: C, D, E, etc.

\subsection{El espacio sin particionar}
\label{sec:orgeed30f4}

Es el espacio que no se ha asignado a ninguna partición.
\end{document}
