% Created 2019-12-16 lun 20:47
% Intended LaTeX compiler: pdflatex
\documentclass[11pt]{article}
\usepackage[utf8]{inputenc}
\usepackage[T1]{fontenc}
\usepackage{graphicx}
\usepackage{grffile}
\usepackage{longtable}
\usepackage{wrapfig}
\usepackage{rotating}
\usepackage[normalem]{ulem}
\usepackage{amsmath}
\usepackage{textcomp}
\usepackage{amssymb}
\usepackage{capt-of}
\usepackage{hyperref}
\usepackage[newfloat]{minted}
\hypersetup{colorlinks=true,linkcolor=black}
\author{Angel Berlanas}
\date{\today}
\title{UD 07 - Exámen (Recuperación)}
\hypersetup{
 pdfauthor={Angel Berlanas},
 pdftitle={UD 07 - Exámen (Recuperación)},
 pdfkeywords={},
 pdfsubject={},
 pdfcreator={Emacs 26.1 (Org mode 9.1.9)}, 
 pdflang={Spanish}}
\begin{document}

\maketitle
\tableofcontents


\section{Instrucciones}
\label{sec:org4b4eb24}

\begin{itemize}
\item Leer \emph{todo} el exámen \emph{de recuperación} hasta el final antes de comenzar a realizar las
acciones.
\item Vais a necesitar la ISO de Xubuntu 19.10 que tenéis disponible en el
Servidor del aula : \emph{172.29.0.254}.
\item No se aceptará otro Sistema Operativo.
\item La instalación ha de ser nueva, \textbf{no podéis} utilizar las antiguas.
\item Mientras se instala podéis ir preparando los comandos y el documento que
entregaréis al final del exámen.
\item \emph{¡MUCHA SUERTE A TOD@S!}.
\end{itemize}

\section{Descripición}
\label{sec:org829d3f5}

A la empresa \emph{Soluciones Eficaces Altamente Inestables} le ha surgido un
competidor, se llama \emph{Pensamiento Profundo}. Se trata de una nueva compañia
que viene apostando fuerte.

\section{Instalación}
\label{sec:orgfe3b6fd}

Debemos instalar una MV con Xubuntu con las siguientes características:

\begin{itemize}
\item 2 CPUs.
\item 2 GB de RAM.
\item 1 Disco Duro de 10GB dinámico.
\item 1 Tarjeta de red en modo \textbf{puente}.
\item En el controlador SATA la \emph{Caché del Anfitrión} ha de estar habilitada.
\end{itemize}

Una vez instalada, instalar el comando \texttt{tree} que nos hará falta para la
entrega, mediante la orden:

\texttt{sudo apt install tree}

\section{Usuarios}
\label{sec:orgb1572a9}

\textbf{Nota: Hasta 3 puntos}

Los usuarios que \emph{finalmente} han de estar configurados en el equipo son los
siguientes (y \textbf{solo} los siguientes), se adjunta tabla con las diferentes
caracterísitcas el usuario:

\begin{center}
\begin{tabular}{lllll}
Usuario & Login & Password & Administrador & Carpeta Personal\\
\hline
Arthur Dent & arthur & pr0t4g0n1st4 & Si & /home/armonia\\
Ford Prefect & prefect & f0rdf0rd & No & /terraqueos/prefect\\
Zaphod Beeblebrox & beeblebrox & pr3s1d3nt3 & Si & /politico/expresidente\\
Tricia McMillan & tricia & tr1ll14n & No & /terraqueos/tricia\\
Marvin & marvin & m3l4nc0l14 & Si & /robots/marvin\\
\end{tabular}
\end{center}

Además existen una serie de grupos \emph{adicionales} cuyos integrantes son los
siguientes:

\begin{center}
\begin{tabular}{lllll}
Usuario/ Grupos & terraqueos & politicos & galacticos & robots\\
\hline
arthur & x &  &  & \\
prefect & x &  & x & \\
beeblebrox &  & x & x & x\\
tricia & x & x & x & \\
marvin &  &  &  & X\\
\end{tabular}
\end{center}

Estos grupos se utilizarán para otorgarles permisos de escritura y lectura a
diferentes carpetas que se encontrarán en diferentes particiones tal y como se
muestra en el siguiente apartado.

\section{Discos Duros y Puntos de Montaje}
\label{sec:org30b3288}

\textbf{Nota: Hasta 3 puntos}

Se añadirán a la máquina virtual los siguientes discos:

\begin{center}
\begin{tabular}{lll}
Disco & Tamaño & Dispositvo\\
\hline
terraqueos.vdi & 2GB & /dev/sdb\\
polgalbots.vdi & 4GB & /dev/sdc\\
compartido.vdi & 2GB & /dev/sdd\\
\end{tabular}
\end{center}


Una vez añadidos se crearán las siguientes particiones y se establecerán los
siguientes puntos de montaje:

\begin{center}
\begin{tabular}{llll}
Partición & Tamaño & Sistema de Ficheros & Punto de Montaje\\
\hline
/dev/sdb1 & 2GB & ext3 & /terraqueos\\
/dev/sdc1 & 2GB & ext4 & /politicos\\
/dev/sdc2 & 2GB & btrfs & /robots\\
/dev/sdd1 & 2GB & ext4 & /compartido\\
\end{tabular}
\end{center}

\section{Permisos de compartido.}
\label{sec:org1eaeb5b}

\textbf{Nota: Hasta 2 puntos}

En la carpeta \texttt{compartido} debe haber una carpeta por cada uno de los \emph{grupos
adicionales} y debemos permitir a los usuarios pertenecientes a dichos grupos
escribir y leer dentro, impidiendo \emph{cualquier otro acceso} a los usuarios que
\textbf{no sean} de ese grupo, excepto al usuario \texttt{marvin} que debe poder realizar
cualquier acción en \textbf{todos} los ficheros y carpetas de \texttt{compartido}.

\section{Pruebas}
\label{sec:orge7d935f}

\textbf{Nota: Hasta 2 puntos}

Se deben realizar las siguientes comprobaciones:

\begin{itemize}
\item Todos los usuarios pueden iniciar sesión gráfica.
\item Todos los usuarios tienen el \texttt{\$HOME} bien establecido.
\item Los permisos de las carpetas personales (\$HOME) son los correctos.
\item Los puntos de montaje son los correctos y se montan en el arranque.
\item Mediante el uso de los comando \texttt{touch} y \texttt{cat} ir pasando por todos los usuarios y
realizando pruebas de creación y lectura de permisos en los diferentes
ficheros y carpetas de \texttt{compartido}.
\end{itemize}


Se deben hacer capturas de pantalla de las siguientes acciones para cada uno
de los usuarios:

\begin{enumerate}
\item Login Gráfico
\item Abrir una terminal y ejecutar las órdenes:
\begin{itemize}
\item \texttt{echo \$HOME}
\item \texttt{groups \$USER}
\item \texttt{touch /compartido/terraqueos/\$USER.datos}
\item \texttt{touch /compartido/politicos/\$USER.datos}
\item \texttt{touch /compartido/galacticos/\$USER.datos}
\item \texttt{touch /compartido/robots/\$USER.datos}
\end{itemize}
\end{enumerate}

\section{Entrega}
\label{sec:orgeb07752}

Debéis presentar en un documento PDF los siguientes \emph{items}.

\begin{itemize}
\item El contenido del fichero \texttt{/etc/fstab} de la MV.
\item La salida del comando \texttt{mount} en la MV.
\item La salida del comando \texttt{tail -n 10 /etc/passwd} en la MV.
\item La salida del comando \texttt{tree -pugf /compartido} en la MV.
\item Una captura de la sesión de \emph{cada uno de los usuarios} donde se muestre que
la carpeta personal es la que está establecida en la \emph{tabla} del exámen y no
\texttt{/home/usuario} (excepto para el usuario \texttt{arthur}).
\item Las capturas de pantalla de las acciones de pruebas del apartado anterior.
\end{itemize}
\end{document}
