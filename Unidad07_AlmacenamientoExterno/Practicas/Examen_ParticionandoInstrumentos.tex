% Created 2019-12-08 dom 10:03
\documentclass[11pt]{article}
\usepackage[utf8]{inputenc}
\usepackage[T1]{fontenc}
\usepackage{fixltx2e}
\usepackage{graphicx}
\usepackage{longtable}
\usepackage{float}
\usepackage{wrapfig}
\usepackage{rotating}
\usepackage[normalem]{ulem}
\usepackage{amsmath}
\usepackage{textcomp}
\usepackage{marvosym}
\usepackage{wasysym}
\usepackage{amssymb}
\usepackage{hyperref}
\tolerance=1000
\usepackage[newfloat]{minted}
\hypersetup{colorlinks=true,linkcolor=black}
\author{Angel Berlanas}
\date{\today}
\title{UD 07 - Exámen}
\hypersetup{
  pdfkeywords={},
  pdfsubject={},
  pdfcreator={Emacs 25.2.2 (Org mode 8.2.10)}}
\begin{document}

\maketitle
\tableofcontents


\section{Instrucciones}
\label{sec-1}

\begin{itemize}
\item Leer \emph{todo} el exámen hasta el final antes de comenzar a realizar las
acciones.
\item Vais a necesitar la ISO de Xubuntu 19.10 que tenéis disponible en el
Servidor del aula : \emph{172.29.0.254}.
\item No se aceptará otro Sistema Operativo.
\item La instalación ha de ser nueva, \textbf{no podéis} utilizar las antiguas.
\item Mientras se instala podéis ir preparando los comandos y el documento que
entregaréis al final del exámen.
\item \emph{¡MUCHA SUERTE A TOD@S!}.
\end{itemize}

\section{Descripición}
\label{sec-2}

La empresa \emph{Soluciones Eficaces Altamente Inestables} nos acaba de contratar,
un nuevo cliente les ha pedido una maqueta de Sistema Operativo para un
estudio de grabación un tanto especial, donde acuden \emph{Youtubers} ,
\emph{Compositores} y \emph{Músicos} de todas las épocas a trabajar (he dicho que era
\emph{un tanto especial}).

Debemos preparar un SO con las necesidades cubiertas para que ellos puedan
probar que nuestra solución les sirve, si les sirve, nos contratarán y
ganaremos un sueldo \emph{Nescafé} para toda la vida.

Además de la instalación, debemos prepararles un pequeño resumen de los
diferentes Discos Duros y de las características que tienen los sistemas de
ficheros FAT32 que los hacen "\emph{tan especiales}".

\section{Instalación}
\label{sec-3}

Debemos instalar una MV con Xubuntu con las siguientes características:

\begin{itemize}
\item 2 CPUs.
\item 2 GB de RAM.
\item 1 Disco Duro de 10GB dinámico.
\item 1 Tarjeta de red en modo \textbf{puente}.
\item En el controlador SATA la \emph{Caché del Anfitrión} ha de estar habilitada.
\end{itemize}

Una vez instalada, instalar el comando \verb~tree~ que nos hará falta para la
entrega, mediante la orden:

\verb~sudo apt install tree~


\section{Usuarios}
\label{sec-4}

\textbf{Nota: Hasta 3 puntos}

Los usuarios que \emph{finalmente} han de estar configurados en el equipo son los
siguientes (y \textbf{solo} los siguientes), se adjunta tabla con las diferentes
caracterísitcas el usuario:

\begin{center}
\begin{tabular}{lllll}
Usuario & Login & Password & Administrador & Carpeta Personal\\
\hline
Armonia & armonia & 4rm0n14 & Si & /home/armonia\\
Jaime & jaltozano & Y0utub3 & No & /youtubers/altozano\\
Luis Angel & debenito & S1gn1f1c4d0 & Si & /musicos/analisis/debenito\\
Anton & bruckner & 0rg4n1st4 & No & /musicos/compositores/bruckner\\
Igor & stravinsky & F1r3b1rd & No & /musicos/compositores/stravinsky\\
Philip & glass & M1n1m4l1st4 & Si & /musicos/pianistas/glass\\
Toshiko & akiyoshi & J4p4nJ4zz & No & /musicos/pianistas/akiyoshi\\
\end{tabular}
\end{center}

Además existen una serie de grupos \emph{adicionales} cuyos integrantes son los
siguientes:

\begin{center}
\begin{tabular}{lllll}
Usuario/ Grupos & analistas & compositores & musicos & youtubers\\
\hline
Armonia & x & x & x & x\\
Jaime & x &  &  & x\\
Luis Angel & x &  & x & \\
Anton &  & x & x & \\
Igor &  & x &  & \\
Philip &  & x & x & \\
Toshiko &  & x & x & \\
\hline
\end{tabular}
\end{center}

Estos grupos se utilizarán para otorgarles permisos de escritura y lectura a
diferentes carpetas que se encontrarán en diferentes particiones tal y como se
muestra en el siguiente apartado.

\section{Discos Duros y Puntos de Montaje}
\label{sec-5}

\textbf{Nota: Hasta 3 puntos}

Se añadirán a la máquina virtual los siguientes discos:

\begin{center}
\begin{tabular}{lll}
Disco & Tamaño & Dispositvo\\
\hline
Musicos.vdi & 6GB & /dev/sdb\\
Youtubers.vdi & 2GB & /dev/sdc\\
Compartido.vdi & 2GB & /dev/sdd\\
\end{tabular}
\end{center}

Una vez añadidos se crearán las siguientes particiones y se establecerán los
siguientes puntos de montaje:

\begin{center}
\begin{tabular}{llll}
Partición & Tamaño & Sistema de Ficheros & Punto de Montaje\\
\hline
/dev/sdb1 & 1GB & ext3 & /musicos/analisis\\
/dev/sdb2 & 2GB & ext4 & /musicos/compositores\\
/dev/sdb3 & 3GB & ext4 & /musicos/pianistas\\
/dev/sdc1 & 2GB & btrfs & /youtubers\\
/dev/sdd1 & 2GB & ext4 & /compartido\\
\end{tabular}
\end{center}

\section{Permisos de compartido.}
\label{sec-6}

\textbf{Nota: Hasta 2 puntos}

En la carpeta \verb~compartido~ debe haber una carpeta por cada uno de los \emph{grupos
adicionales} y debemos permitir a los usuarios pertenecientes a dichos grupos
escribir y leer dentro, impidiendo \emph{cualquier otro acceso} a los usuarios que
\textbf{no sean} de ese grupo, excepto al usuario \emph{armonia} que debe poder realizar
cualquier acción en \textbf{todos} los ficheros y carpetas de \verb~compartido~.

\section{Pruebas}
\label{sec-7}

\textbf{Nota: Hasta 2 puntos}

Se deben realizar las siguientes comprobaciones:

\begin{itemize}
\item Todos los usuarios pueden iniciar sesión gráfica.
\item Todos los usuarios tienen el \verb~$HOME~ bien establecido.
\item Los permisos son los correctos.
\item Los puntos de montaje son los correctos y se montan en el arranque.
\item Mediante el uso de los comando \verb~touch~ y \verb~cat~ ir pasando por todos los usuarios y
realizando pruebas de creación y lectura de permisos en los diferentes
ficheros y carpetas de \verb~compartido~.
\end{itemize}


\section{Petición especial}
\label{sec-8}

\textbf{Nota: Hasta 2 puntos}

La tarea de formatear discos es algo muy importante y que debemos realizar con
muchísima atención, sin embargo muchas veces veremos que resulta una tarea
tediosa si lo tenemos que hacer en centenares de discos la misma
operación. 

Hemos consultado con el \emph{Oráculo} y nos ha dicho que existe \emph{al menos 1}
script que formatea las particiones de un dispositivo dado.

El \emph{script} es este:

\begin{minted}[]{bash}
#!/bin/bash
(
echo o # Create a new empty DOS partition table
echo n # Add a new partition
echo p # Primary partition
echo 1 # Partition number
echo   # First sector (Accept default: 1)
echo   # Last sector (Accept default: varies)
echo w # Write changes
) | sudo fdisk /dev/sdX

exit 0
\end{minted}

Se piden dos tareas:

\begin{itemize}
\item Explicar \emph{qué hace} este script, es decir \emph{cómo funciona}.
\item Modificarlo para que adapte a tus necesidades en el formateo del disco : \emph{Compartido.vdi}.
\end{itemize}


\section{Entrega}
\label{sec-9}

Debéis presentar en un documento PDF los siguientes \emph{items}.

\begin{itemize}
\item El contenido del fichero \verb~/etc/fstab~ de la MV.
\item La salida del comando \verb~mount~ en la MV.
\item La salida del comando \verb~cat /etc/passwd~ en la MV.
\item La salida del comando \verb~tree -pugf /compartido~ en la MV.
\item Una captura de la sesión de \emph{cada uno de los usuarios} donde se muestre que
la carpeta personal es la que está establecida en la \emph{tabla} del exámen y no
\verb~/home/usuario~ (excepto para el usuario \verb~armonia~).
\end{itemize}
% Emacs 25.2.2 (Org mode 8.2.10)
\end{document}
