% Created 2020-01-06 lun 12:01
% Intended LaTeX compiler: pdflatex
\documentclass[11pt]{article}
\usepackage[utf8]{inputenc}
\usepackage[T1]{fontenc}
\usepackage{graphicx}
\usepackage{grffile}
\usepackage{longtable}
\usepackage{wrapfig}
\usepackage{rotating}
\usepackage[normalem]{ulem}
\usepackage{amsmath}
\usepackage{textcomp}
\usepackage{amssymb}
\usepackage{capt-of}
\usepackage{hyperref}
\usepackage[newfloat]{minted}
\hypersetup{colorlinks=true,linkcolor=black}
\author{Angel Berlanas}
\date{\today}
\title{UD08 - Tarjetas}
\hypersetup{
 pdfauthor={Angel Berlanas},
 pdftitle={UD08 - Tarjetas},
 pdfkeywords={},
 pdfsubject={},
 pdfcreator={Emacs 26.3 (Org mode 9.1.9)}, 
 pdflang={English}}
\begin{document}

\maketitle
\tableofcontents


\section{Introducción}
\label{sec:org1d23c1f}

Conceptos sobre la ampliación y modificación.
El modelo de PC y Amstrad.
Conexiones en el modelo de Von Neumman.

\section{Conexiones a la placa base}
\label{sec:org2cb0b6c}

Historia de las conexiones
AGP
PCI Express

\section{PCIExpress}
\label{sec:org629c844}

PCI Express está organizado en lanes. Cada lane tiene un conjunto
independiente de pines de transmisión y recepción, y los datos pueden enviarse
en ambas direcciones simultáneamente. Y aquí es donde las cosas se vuelven
engañosas. El ancho de banda en una sola dirección para un solo lane PCIe 1.0
(x1) es de 250 MB/s, pero debido a que puede enviar y recibir 250 MB/s al
mismo tiempo a Intel le gusta indicar el ancho de banda disponible para una
ranura PCIe 1.0 x1 como 500 MB / s . Si bien ese es el ancho de banda total
agregado disponible para una sola ranura, solo puedes alcanzar esa cifra de
ancho de banda si estás leyendo y escribiendo al mismo tiempo.


\begin{itemize}
\item Las conexiones ‘PCIe x1’ tienen un lane de datos
\item Las conexiones ‘PCIe x4’ tienen cuatro lanes de datos
\item Las conexiones ‘PCIe x8’ tienen ocho lanes de datos
\item Las conexiones ‘PCIe x16’ tienen dieciséis lanes de datos
\item Las conexiones ‘PCIe x32’ tienen treinta y dos lanes de datos (actualmente, son muy raras)
\end{itemize}

\begin{center}
\begin{tabular}{lllll}
Velocidad & PCI-e 1.0 & PCI-e 2.x & PCI-e 3.0 & PCI-e 4.x\\
\hline
x1 & 250MB/s & 500MB/s & 985MB/s & 1969MB/s\\
x4 & 1000MB/s & 2000MB/s & 3940MB/s & 7876MBs\\
x8 & 2000MB/s & 4000MB/s & 7880MB/s & 15752MB/s\\
x16 & 4000MB/s & 8000MB/s & 15760MB/s & 31504MB/s\\
\end{tabular}
\end{center}


\section{Tipos de Tarjeta}
\label{sec:org66f2c42}
\end{document}
