% Created 2020-01-14 mar 06:25
% Intended LaTeX compiler: pdflatex
\documentclass[11pt]{article}
\usepackage[utf8]{inputenc}
\usepackage[T1]{fontenc}
\usepackage{graphicx}
\usepackage{grffile}
\usepackage{longtable}
\usepackage{wrapfig}
\usepackage{rotating}
\usepackage[normalem]{ulem}
\usepackage{amsmath}
\usepackage{textcomp}
\usepackage{amssymb}
\usepackage{capt-of}
\usepackage{hyperref}
\usepackage[newfloat]{minted}
\hypersetup{colorlinks=true,linkcolor=black}
\author{Angel Berlanas}
\date{\today}
\title{UD 08 - Ejercicios lspci}
\hypersetup{
 pdfauthor={Angel Berlanas},
 pdftitle={UD 08 - Ejercicios lspci},
 pdfkeywords={},
 pdfsubject={},
 pdfcreator={Emacs 26.3 (Org mode 9.1.9)}, 
 pdflang={Spanish}}
\begin{document}

\maketitle
\tableofcontents


\section{Introducción}
\label{sec:org6c5e7ea}

Acabamos de entrar en la empresa \emph{Profesionales Responsables}, estamos en un
contrato de prueba, esta mañana a primera hora ha venido un cliente y nos ha
pedido una serie de \texttt{scripts} que le ayuden para identificar una serie de
\emph{drivers} que tiene en sus sistemas.

Está teniendo problemas con algunos y necesita saber cuáles \emph{blacklistear}.

Debemos entregar un \texttt{script} por cada uno de los ejercicios que se muestran a
continuación.

Para ello necesitais de una pequeña \emph{masterclass} que os acaba de dar vuestro
compañero. 

PD: La numeración y nombres de los scripts son importantes.

\section{script01IntelAmd.sh}
\label{sec:orgc22d52b}

Crear un Script que cuando se ejecute, pregunte al usuario si quiere listar
los dispositivos conectados a los puertos PCI que son de la marca intel o amd.

Una vez preguntado, listar \emph{solo} aquellos que sean de esa marca y a
continuación se muestre el número de dispositivos.

\section{script02PCIBridge.sh}
\label{sec:org0715e74}

Crear un Script que nos indique cuantos dispositivos de tipo \texttt{PCI bridge} tenemos y
\emph{además} los liste.

\section{script03ModuloKernel.sh}
\label{sec:org33831ae}

Crear un Script que nos pregunte si Tarjeta de Red o de Audio y nos muestre el
módulo del kernel asociado a dicha tarjeta pci.

\section{script04RedBasico.sh}
\label{sec:orgbc6c00e}

Crear un script que muestre el nombre de los diferentes dispositivos de red
que tenga nuestra máquina. A continuación nos debe preguntar por uno de ellos
y nosotros podremos introducir su nombre. Una vez introducido, mostrará la IP
en su versión IPv4 e IPv6.

\section{script05AudioNoTanBasico.sh}
\label{sec:orgd34817e}

Crear un script que muestre en porcentaje el volumen del audio del altavoz
maestro.

Ejemplo de salida:

\begin{minted}[]{bash}

smx@cliente:~$./script05_audio_notanbasico.sh
23%
smx@cliente:~$

\end{minted}
\end{document}
