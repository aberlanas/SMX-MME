% Created 2020-02-17 lun 20:02
% Intended LaTeX compiler: pdflatex
\documentclass[11pt]{article}
\usepackage[utf8]{inputenc}
\usepackage[T1]{fontenc}
\usepackage{graphicx}
\usepackage{grffile}
\usepackage{longtable}
\usepackage{wrapfig}
\usepackage{rotating}
\usepackage[normalem]{ulem}
\usepackage{amsmath}
\usepackage{textcomp}
\usepackage{amssymb}
\usepackage{capt-of}
\usepackage{hyperref}
\usepackage[newfloat]{minted}
\hypersetup{colorlinks=true,linkcolor=black}
\author{Angel Berlanas Vicente}
\date{\today}
\title{UD09 - Repaso - Placas Base}
\hypersetup{
 pdfauthor={Angel Berlanas Vicente},
 pdftitle={UD09 - Repaso - Placas Base},
 pdfkeywords={},
 pdfsubject={},
 pdfcreator={Emacs 26.3 (Org mode 9.1.9)}, 
 pdflang={Spanish}}
\begin{document}

\maketitle
\tableofcontents


\section{Aviso navegantes}
\label{sec:orgb77d332}

Leed atentamente los enunciados de las actividades y buscad soluciones que se
adapten a las necesidades.

\section{Pregunta 01 : Placas Base.}
\label{sec:orgcdd6b4c}

La placa base es uno de los componentes más importantes de nuestro ordenador,
sin embargo se le presta poca atención más allá de que sea capaz de soportar
los diferentes componentes que van \emph{sobre} ella. Esto a veces da lugar a un
fenómeno muy habitual en el mundo de los computadores que es el \emph{Cuello de
Botella}. Redacta un pequeño texto en el que expliques de manera sencilla en
que consiste dicho problema y sobre que tipo de recursos se suele dar.

\section{Pregunta 02 : SouthBridge}
\label{sec:orgf347f0d}

En el puente sur \emph{(SouthBridge)} están una conectados una serie de
dispositivos. ¿Qué tipos de dispositivos son?. 

\section{Pregunta 03 : dmidecode}
\label{sec:orga0c3e66}

Si hechábais en falta los scripts\ldots{}no hay problema aquí tenéis unos pocos
para que no perdamos la costumbre.

Para obtener la nota máxima por cada uno de los scripts deben ser ejecutados y
que devuelvan exáctamente lo mismo que se muestra (tened en cuenta que puede
ser diferente en vuestras máquinas, pero no vale hacer un \texttt{echo COSAS}).

Probad los scripts, aseguraos que muestran \emph{exactamente lo que pido}. Es muy
sencillo obtener un 10 y muy facil un 0.

Pistas: \emph{Todos los scripts se realizan parseando la salida de \texttt{dmidecode} y}
\emph{ejecutando operaciones sobre esos datos obtenidos.}.

Para la ejecución de esto necesitaréis o un \textbf{\textbf{GNU/LinuX}} instalado o la
Máquina Virtual.

Los nombres de los scripts son los que aparecen como título de los diferentes apartados.

\subsection{dmidecode01.sh}
\label{sec:orgf17f3ef}

Debe mostrar el voltaje al que trabaja vuestro procesador:

\begin{center}
\begin{tabular}{ll}
\textbf{\textbf{Ejecución}} : & \texttt{sudo ./dmidecode01.sh}\\
\textbf{\textbf{Salida}}: & \texttt{Voltios: 1.3 V}.\\
\end{tabular}
\end{center}

\subsection{dmidecode02.sh}
\label{sec:org5a55dbf}

Debe mostrar el \emph{Vendor} de la BIOS.

\begin{center}
\begin{tabular}{ll}
\textbf{\textbf{Ejecución}} : & \texttt{sudo ./dmidecode02.sh}\\
\textbf{\textbf{Salida}}: & \texttt{Vendedor: American Megatrends Inc}.\\
\end{tabular}
\end{center}

\subsection{dmidecode03.sh}
\label{sec:org8d85d88}

Debe indicar si soporta UEFI.

\begin{center}
\begin{tabular}{ll}
\textbf{\textbf{Ejecución}} : & \texttt{sudo ./dmidecode03.sh}\\
\textbf{\textbf{Salida}}: & \texttt{Soporta UEFI : Si}.\\
\end{tabular}
\end{center}

\subsection{dmidecode04.sh}
\label{sec:org96c164e}

Debe indicar de cada uno de los Slots de PCIExpress x1 que voltaje provee y
si está disponible o no.

\begin{center}
\begin{tabular}{ll}
\textbf{\textbf{Ejecución}} : & \texttt{sudo ./dmidecode04.sh}\\
\textbf{\textbf{Salida}}: & \texttt{x1\_1 : Disponible : 3,3 V}.\\
 & \texttt{x1\_2 : No disponible: 3,3 V}.\\
\end{tabular}
\end{center}


\section{Pregunta 04 : BIOS y UEFI.}
\label{sec:org6a7b66f}

¿Recordáis el texto en inglés que hablaba del arranque de un equipo?
Indica en el proceso que vas a ver a continuación si hubiera algún error:

\begin{center}
\begin{tabular}{rl}
Paso & Descripción\\
1 & Equipo se pone en marcha.\\
2 & La BIOS comprueba que disco duro es el primero.\\
3 & Se ejecuta el POST.\\
4 & Una vez seleccionado el disco duro, se busca el sector de arranque.\\
5 & En el caso de las UEFIs se permiten tamaños más grandes de particiones\\
6 & No se puede modificar el siguiente reinicio desde el Sistema Operativo arrancado.\\
\end{tabular}
\end{center}

En caso de que hubiera algún error puede estar en dos sitios: Por un lado en
el \emph{orden} de la ejecución y por otro en la \emph{Descripción}. Reescribe los pasos
indicando qué matiz debe corregirse para que sea cierto. Tenéis la solución en
el artículo \^{}\_\^{}.
\end{document}
